%-----------------------------------------
% Banded Myers
%-----------------------------------------

\newcommand{\parallelogram}[4] { % params: start x, start y, end x, diagonal width
  \fill[color=black, opacity=0.2] (#1,#2) -- (#3,#2+#3-#1) -- (#3,#2+#3-#1+#4) -- (#1,#2+#4) -- cycle;
}
\newcommand{\pix}[3] { \draw[fill=#3,color=#3] (#1,#2) circle (2pt); }
\newcommand{\diagline}[3] {
  \draw[line width=1pt](#1+.5,#2+.5)--(#1+#3+.5,#2+#3+.5);
}
\newcommand{\diaglineDotted}[3] {
  \draw[line width=0.8pt,densely dotted](#1+.5,#2+.5)--(#1+#3+.5,#2+#3+.5);
}
\newcommand{\diaglineB}[3] {
  \draw[line width=1pt](#1+.5,#2+.5)--(#1+#3+.5,#2+#3+.5);
}
\newcommand{\horizline}[3] {
  \draw[line width=1pt](#1+.5,#2+.5)--(#1+#3+.5,#2+.5);
}
\newcommand{\vertline}[3] {
  \draw[line width=1pt](#1+.5,#2+.5)--(#1+.5,#2+#3+.5);
}
\newcommand{\bandedDotPara}[4]
{ %x,y,d,w
	\foreach \val in {1,...,#4}
	{
		\ifnum\val<#3
			\fill[fill=DPRecurseColor] (#1+\val,#2+1) rectangle (#1+\val + 1, #2+\val + 1);
		\else
			\ifnum\val<\numexpr #4-1
				\fill[fill=DPRecurseColor] (#1+\val,#2+\val-#3+1) rectangle (#1+\val+1,#2+\val+1);
			\else
				\fill[fill=DPRecurseColor] (#1+\val,#2+\val-#3+1) rectangle (#1+\val+1,#2+#4-1);
			\fi
		\fi
	}
}

\newcommand{\bandedStepsBackground}
{
	% paralelogram and grid
	\begin{pgfscope}
		\tikzstyle{grid lines}=[gray,densely dotted,line width=0.3pt]
		\pgftransformshift{\pgfpoint{-3cm}{0cm}}
		\clip (3,-7.3) rectangle (6.3, 7.3);
		\bandedDotPara{0}{0}{11}{19};
		\fill[fill=DPInitColor] (3,0) rectangle (4,4);
		\fill[fill=DPInitColor] (3,0) rectangle (11,1);
%		\draw[black, densely dotted] (12,-1) rectangle (13, 20);
		\draw[style=grid lines](0,0) grid +(20,20);
		\draw[gray] (3,0) rectangle (19, 13);
	\end{pgfscope}
}

\newcommand{\bandedInitBackground}
{
	% paralelogram and grid
	\begin{pgfscope}
		\tikzstyle{grid lines}=[gray,densely dotted,line width=0.3pt]
		\pgftransformshift{\pgfpoint{-3cm}{0cm}}
		\clip (3,-7.3) rectangle  (15.3, 7.3);
		\bandedDotPara{0}{0}{11}{19};
		\fill[fill=DPInitColor] (3,-7) rectangle (4,-6);
		\fill[fill=DPInitColor] (3,-6) rectangle (5,-5);
		\fill[fill=DPInitColor] (3,-5) rectangle (6,-4);
		\fill[fill=DPInitColor] (3,-4) rectangle (7,-3);
		\fill[fill=DPInitColor] (3,-3) rectangle (8,-2);
		\fill[fill=DPInitColor] (3,-2) rectangle (9,-1);
		\fill[fill=DPInitColor] (3,-1) rectangle (10,0);
		\fill[fill=DPInitColor] (3,0) rectangle (11,1);
		\fill[fill=DPInitColor] (3,0) rectangle (4,4);
		\fill[fill=DPInitColorDark] (3,3) rectangle (4,4);
		\fill[fill=DPRecurseColorDark] (4,4) rectangle (5,5);
		\fill[fill=DPRecurseColorDark] (5,5) rectangle (6,6);
		\fill[fill=DPRecurseColorDark] (6,6) rectangle (7,7);
		\fill[fill=DPRecurseColorDark] (7,7) rectangle (8,8);
		\fill[fill=DPRecurseColorDark] (8,8) rectangle (9,9);
		\fill[fill=DPRecurseColorDark] (9,9) rectangle (10,10);
		\fill[fill=DPRecurseColorDark] (10,10) rectangle (11,11);
		\fill[fill=DPRecurseColorDark] (11,11) rectangle (12,12);
		\fill[fill=DPRecurseColorDark] (12,12) rectangle (19,13);
%		\draw[black, densely dotted] (12,-1) rectangle (13, 20);
		\draw[style=grid lines](0,-7) grid +(20,20);
		\draw[black] (3,0) rectangle (19, 13);
	\end{pgfscope}
}
