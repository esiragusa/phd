%-----------------------------------------
% Print Strings and Tables
%-----------------------------------------

\newcommand {\PrintChar} [3]
{
	\node[anchor=base] at (#1 + .5, #2 + \ifdefined\CharYOfs \CharYOfs \else + .7 \fi) [char] { \ifdefined\CharFont \CharFont \else \Large\ttfamily \fi {#3} };
}

\newcommand {\PrintIndex} [3]
{
	\node[anchor=base] at (#1 + .5, #2 - .25) [position,color=NumberColor] { \sffamily\footnotesize{#3} };
}

\newcommand {\PrintCharWithIndex} [4]
{
	\PrintChar{#1}{#2}{#3}
	\PrintIndex{#1}{#2}{#4}
}

\newcounter {PrintString@Index}
\def\reallyempty{}
\newcommand {\PrintIndices} [3]
{
	\foreach[count=\i from 0] \val in {#3}
	{
		\setcounter {x} {\i}
		\PrintIndex {#1 + \arabic{x}} {#2} {\val} {\tt{\arabic{x}}}
	}
}
\newcommand {\PrintString} [4]
{
	\node at (#1 + .2, #2 + .75) [title] { \LARGE { #3 } };
	\foreach[count=\i from 0] \val in {#4}
	{
		\setcounter {x} {\i}
		\PrintChar {#1 + \arabic{x}} {#2} {\val} {\arabic{x}}
%		\draw (#1 + \arabic{x},#2) grid +(1, 1);
		\draw[black] (#1 + \arabic{x},#2) rectangle (#1 + \arabic{x} + 1,#2 + 1);
	}
%	\draw (#1,#2) grid +(\arabic{x}, 1);
}

\newcommand {\PrintStringWithIndex} [4]
{
	\setcounter {PrintString@Index} {0}
	\node at (#1 + .2, #2 + .75) [title] { \LARGE { #3 } };
	\foreach[count=\i from 0] \val in {#4}
	{
		\setcounter {x} {\i}
		\ifx\val\reallyempty
		\else
			\PrintCharWithIndex {#1 + \arabic{x}} {#2} {\val} {\arabic{PrintString@Index}}
			\addtocounter {PrintString@Index} {1}
%			\draw (#1 + \arabic{x},#2) grid +(1, 1);
			\draw[black] (#1 + \arabic{x},#2) rectangle (#1 + \arabic{x} + 1,#2 + 1);
		\fi
	}
%	\draw (#1,#2) grid +(\arabic{x}, 1);
}

\newcounter{tokcnt}
\newcommand {\PrintVString} [3]
{
	\strlenstore{#3}{tokcnt}
	\foreach \val in {1,2,...,\value{tokcnt}}
	{
		\FPmul{\ofsY}{\val}{.7}
		\PrintChar {#1} {#2 - 0.7 + \ofsY} {\color{NumberColor}\substr{#3}{\val}{1}}
	}
}

\newcommand {\PrintStringOnly} [3]
{
	\foreach[count=\i from 0] \val in {#3}
	{
		\setcounter {x} {\i}
		\PrintChar {#1 + \arabic{x}} {#2} {\val} {\arabic{x}}
	}
}

\newcommand {\PrintVStringOnly} [3]
{
	\foreach[count=\i from 0] \val in {#3}
	{
		\setcounter {x} {\i}
		\PrintChar {#1} {#2 + \arabic{x}} {\val} {\arabic{x}}
	}
}

\newcommand {\PrintDots} [3]
{
	\foreach[count=\i from 0] \val in {#3}
	{
		\setcounter {x} {\i}
		\ifthenelse{\equal{\val}{0}}
		{
			\PrintChar {#1 + \arabic{x}} {#2} {$\cdot$} {\arabic{x}}
		}
		{
			\PrintChar {#1 + \arabic{x}} {#2} {$\bullet$} {\arabic{x}}
		}
	}
}
