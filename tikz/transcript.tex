\tikzstyle{n}=[inner sep=0pt, minimum size=10pt, align=center]
\tikzstyle{e}=[-latex, thin]
\tikzstyle{m}=[draw, shape=circle, clabel, pos=0.4, align=center, inner sep=0pt, minimum size=8pt, font=\tiny]
\tikzstyle{t}=[draw, shape=circle, clabel, pos=0.4, align=center, inner sep=0pt, minimum size=8pt, font=\tiny, fill=LightGray]
\tikzstyle{frame}=[draw, rectangle, thin, inner sep=0pt]
\tikzstyle{covered}=[draw, rectangle, thin, inner sep=0pt, fill=LightGray]
\tikzstyle{tape}=[fill=black]

\tikzstyle{strike}=[-, style=double, thin, decorate, decoration=zigzag]
\tikzstyle{break}=[-, thin, decorate, decoration=zigzag]
\tikzstyle{wave}=[-, thin, decorate, decoration={snake, segment length=2.5mm, amplitude=0.4mm}]
\tikzstyle{line}=[-, thin]
\tikzstyle{gap}=[-, style=double, thin, double distance=0.3mm]

\newcommand{\semitranscript}[5]
{
    \foreach[count=\i] \r/\t/\g in {#2}
    {
    	\node[n] (read_\i) at (0.4*\i,#3*0.4) {\r};
		\node[n] (genome_\i)  at (0.4*\i,-1) {\g};
		\ifthenelse{\equal{\t}{$ $}}
	    {
%			\draw[e] (read_\i.south)+(0,-#3*0.4) -- (genome_\i) node[m] (transcript_\i) {\t};
		}
		{
			\draw[e] (read_\i.south)+(0,-#3*0.4) -- (genome_\i) node[m] (transcript_\i) {\t};
		}
    }
    
    \begin{pgfonlayer}{background} 
		\draw[tape] ([xshift=0.1cm, yshift=-0.01cm]transcript_#4.north west)
				rectangle
					([xshift=-0.1cm, yshift=0.01cm]transcript_#5.south east) ;
	\end{pgfonlayer}
}

\newcommand{\transcript}[3]
{
    \semitranscript{#1}{#2}{#3}{1}{#1}
}

\newcommand{\band}[1]
{
	\draw[line] (genome_1.north west) -- (genome_#1.north east) ;
	\draw[line] (genome_1.south west) -- (genome_#1.south east) ;
	\draw[wave] (genome_1.north west) -- (genome_1.south west) ;
	\draw[wave] (genome_#1.north east) -- (genome_#1.south east) ;
}

% Now used to draw a frame around the pattern.
\newcommand{\seed}[3]
{
	\ifthenelse{\equal{#3}{0}}
    {
	    \begin{pgfonlayer}{background} 
    		\draw[covered] (read_#1.north west) rectangle (read_#2.south east) ;
		\end{pgfonlayer}
    }

	\node[frame] (read_rect) [transform shape, fit = (read_#1) (read_#2)] {};
}

% Contiguous q-gram ####
\newcommand{\qgram}[4]
{
	\pgfmathsetmacro\ypos{0.4*4}
	
    \foreach[count=\i, evaluate=\i as \pos using {(#1-1+\i)}] \r in {#2}
    {
    	\node[n] (qgram_#1_\i) at (0.4*\pos,\ypos-0.4*#3) {\phantom{\r}};
	}
	\ifthenelse{\equal{#4}{0}}
    {
		\node[frame] (read_rect) [transform shape, fit = (qgram_#1_1) (qgram_#1_4)] {};
	}
    {
		\node[frame, fill=LightGray] (read_rect) [transform shape, fit = (qgram_#1_1) (qgram_#1_4)] {};
	}
}

% Approximate seed
\newcommand{\qgrama}[5]
{
	\pgfmathsetmacro\ypos{0.4}

    \foreach[count=\i, evaluate=\i as \pos using {(#1-1+\i)}] \r in {#2}
    {
    	\node[n] (qgram_#1_\i) at (0.4*\pos,\ypos-0.4*#3) {\phantom{\r}};
	}
	\ifthenelse{\equal{#5}{0}}
    {
		\node[frame] (read_rect) [transform shape, fit = (qgram_#1_1) (qgram_#1_#4)] {};
	}
    {
		\node[frame, fill=LightGray] (read_rect) [transform shape, fit = (qgram_#1_1) (qgram_#1_#4)] {};
	}
}

% Exact seed
\newcommand{\qgramo}[4]
{
	\qgrama{#1}{#2}{#3}{4}{#4}
}

% Exact seed spanning an insertion
\newcommand{\qgrami}[3]
{
    \foreach[count=\i, evaluate=\i as \pos using {(#1-1+\i)}] \r in {#2}
    {
    	\node[n] (qgram_#1_\i) at (0.4*\pos,0.4-0.4*#3) {\phantom{\r}};
	}

	\draw[line] (qgram_#1_1.north west) -- (qgram_#1_2.north west) ;
	\draw[line] (qgram_#1_1.north west) -- (qgram_#1_1.south west) ;
	\draw[line] (qgram_#1_1.south west) -- (qgram_#1_2.south west) ;
	\draw[break] (qgram_#1_1.north east) -- (qgram_#1_2.south west) ;

	\draw[line] (qgram_#1_2.north east) -- (qgram_#1_5.north east) ;
	\draw[line] (qgram_#1_5.north east) -- (qgram_#1_5.south east) ;
	\draw[line] (qgram_#1_2.south east) -- (qgram_#1_5.south east) ;
	\draw[break] (qgram_#1_2.north east) -- (qgram_#1_2.south east) ;
}

% Approximate seed spanning an insertion
\newcommand{\qgramai}[3]
{
    \foreach[count=\i, evaluate=\i as \pos using {(#1-1+\i)}] \r in {#2}
    {
    	\node[n] (qgram_#1_\i) at (0.4*\pos,0.4-0.4*#3) {\phantom{\r}};
	}

	\draw[line] (qgram_#1_1.north west) -- (qgram_#1_2.north west) ;
	\draw[line] (qgram_#1_1.north west) -- (qgram_#1_1.south west) ;
	\draw[line] (qgram_#1_1.south west) -- (qgram_#1_2.south west) ;
	\draw[break] (qgram_#1_1.north east) -- (qgram_#1_2.south west) ;

	\draw[line] (qgram_#1_2.north east) -- (qgram_#1_9.north east) ;
	\draw[line] (qgram_#1_9.north east) -- (qgram_#1_9.south east) ;
	\draw[line] (qgram_#1_2.south east) -- (qgram_#1_9.south east) ;
	\draw[break] (qgram_#1_2.north east) -- (qgram_#1_2.south east) ;
}

% Gapped q-gram ##-#
\newcommand{\qgramg}[4]
{
    \foreach[count=\i, evaluate=\i as \pos using {(#1-1+\i)}] \r in {#2}
    {
    	\node[n] (qgram_#1_\i) at (0.4*\pos,2.0-0.4*#3) {\phantom{\r}};
	}

	\ifthenelse{\equal{#4}{0}}
    {
		\node[frame] (read_rect) [transform shape, fit = (qgram_#1_1)] {};
		\node[frame] (read_rect) [transform shape, fit = (qgram_#1_3) (qgram_#1_5)] {};
	}
	{
		\node[frame, fill=LightGray] (read_rect) [transform shape, fit = (qgram_#1_1)] {};
		\node[frame, fill=LightGray] (read_rect) [transform shape, fit = (qgram_#1_3) (qgram_#1_5)] {};
	}
	
	\draw[gap] (qgram_#1_1) -- (qgram_#1_3) ;
}

% Gapped q-gram #-###
\newcommand{\qgramgrev}[4]
{
    \foreach[count=\i, evaluate=\i as \pos using {(#1-1+\i)}] \r in {#2}
    {
    	\node[n] (qgram_#1_\i) at (0.4*\pos,2.0-0.4*#3) {\phantom{\r}};
	}

	\ifthenelse{\equal{#4}{0}}
    {
		\node[frame] (read_rect) [transform shape, fit = (qgram_#1_1) (qgram_#1_3)] {};
		\node[frame] (read_rect) [transform shape, fit = (qgram_#1_5)] {};
	}
	{
		\node[frame, fill=LightGray] (read_rect) [transform shape, fit = (qgram_#1_1) (qgram_#1_3)] {};
		\node[frame, fill=LightGray] (read_rect) [transform shape, fit = (qgram_#1_5)] {};
	}
	\draw[gap] (qgram_#1_3) -- (qgram_#1_5) ;
}