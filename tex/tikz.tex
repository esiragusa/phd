%-----------------------------------------
% TiKZ Macros
%-----------------------------------------

%
% Suffix tree labels
%

\tikzstyle{root}=[circle,fill,color=gray,inner sep=1.5pt]
\tikzstyle{inner}=[circle,draw,color=gray,inner sep=1.5pt]
\tikzstyle{implicit}=[circle,fill,inner sep=0.2pt]
\tikzstyle{leaf}=[rectangle,color=black,draw,inner sep=1.5pt]
\tikzstyle{pf}=[->,very thin,color=gray,text=mycolor1high]
\tikzstyle{kpf}=[very thin]
\tikzstyle{clabel}=[fill=white,inner sep=1pt,font=\small\tt]
% In Ruby, we do
% 1.upto(9) do |i| x='\newcommand{\label'+(i+64).chr+'}['+i.to_s+']{' ; 1.upto(i) do |j| x<<'node[pos='+(((j.to_f/(i+1))*1000).floor/1000.0).to_s+']{\texttt{#'+j.to_s+'}} ' end; x<<'}'; puts x end
% to generate:
\newcommand{\labelA}[1]{node[clabel,pos=0.5]{#1}}
\newcommand{\labelB}[2]{node[clabel,pos=0.333]{#1} node[clabel,pos=0.666]{#2} }
\newcommand{\labelC}[3]{node[clabel,pos=0.25]{#1} node[clabel,pos=0.5]{#2} node[clabel,pos=0.75]{#3} }
\newcommand{\labelD}[4]{node[clabel,pos=0.2]{#1} node[clabel,pos=0.4]{#2} node[clabel,pos=0.6]{#3} node[clabel,pos=0.8]{#4} }
\newcommand{\labelE}[5]{node[clabel,pos=0.166]{#1} node[clabel,pos=0.333]{#2} node[clabel,pos=0.5]{#3} node[clabel,pos=0.666]{#4} node[clabel,pos=0.833]{#5} }
\newcommand{\labelF}[6]{node[clabel,pos=0.142]{#1} node[clabel,pos=0.285]{#2} node[clabel,pos=0.428]{#3} node[clabel,pos=0.571]{#4} node[clabel,pos=0.714]{#5} node[clabel,pos=0.857]{#6} }
\newcommand{\labelG}[7]{node[clabel,pos=0.125]{#1} node[clabel,pos=0.25]{#2} node[clabel,pos=0.375]{#3} node[clabel,pos=0.5]{#4} node[clabel,pos=0.625]{#5} node[clabel,pos=0.75]{#6} node[clabel,pos=0.875]{#7} }
\newcommand{\labelH}[8]{node[clabel,pos=0.111]{#1} node[clabel,pos=0.222]{#2} node[clabel,pos=0.333]{#3} node[clabel,pos=0.444]{#4} node[clabel,pos=0.555]{#5} node[clabel,pos=0.666]{#6} node[clabel,pos=0.777]{#7} node[clabel,pos=0.888]{#8} }
\newcommand{\labelI}[9]{node[clabel,pos=0.1]{#1} node[clabel,pos=0.2]{#2} node[clabel,pos=0.3]{#3} node[clabel,pos=0.4]{#4} node[clabel,pos=0.5]{#5} node[clabel,pos=0.6]{#6} node[clabel,pos=0.7]{#7} node[clabel,pos=0.8]{#8} node[clabel,pos=0.9]{#9} }
% this ;-)

%
% Print Strings and Tables
%

\newcommand {\PrintChar} [3]
{
	\node[anchor=base] at (#1 + .5, #2 + \ifdefined\CharYOfs \CharYOfs \else + .7 \fi) [char] { \ifdefined\CharFont \CharFont \else \Large\ttfamily \fi {#3} };
}
\newcommand {\PrintIndex} [3]
{
	\node[anchor=base] at (#1 + .5, #2 - .25) [position,color=NumberColor] { \sffamily\footnotesize{#3} };
}
\newcommand {\PrintCharWithIndex} [4]
{
	\PrintChar{#1}{#2}{#3}
	\PrintIndex{#1}{#2}{#4}
}

\newcounter{wotdFig}
\newcounter{skewFig}
\newcounter {x}
\newcounter{sy}
\setcounter{sy}{3}
\newcounter{ty}
\setcounter{ty}{-5}
\newboolean{wotdorig} %Deklaration
\setboolean{wotdorig}{false} %Zuweisung

\newcounter {PrintString@Index}
\def\reallyempty{}
\newcommand {\PrintIndices} [3]
{
	\setcounter {x} {0}
	\foreach \val in {#3}
	{
		\PrintIndex {#1 + \arabic{x}} {#2} {\val} {\tt{\arabic{x}}}
		\addtocounter {x} {1}
	}
}
\newcommand {\PrintString} [4]
{
	\setcounter {x} {0}
	\node at (#1 + .2, #2 + .75) [title] { \LARGE { #3 } };
	\foreach \val in {#4}
	{
		\PrintChar {#1 + \arabic{x}} {#2} {\val} {\arabic{x}}
%		\draw (#1 + \arabic{x},#2) grid +(1, 1);
		\draw[black] (#1 + \arabic{x},#2) rectangle (#1 + \arabic{x} + 1,#2 + 1);
		\addtocounter {x} {1}
	}
%	\draw (#1,#2) grid +(\arabic{x}, 1);
}
\newcommand {\PrintStringWithIndex} [4]
{
	\setcounter {x} {0}
	\setcounter {PrintString@Index} {0}
	\node at (#1 + .2, #2 + .75) [title] { \LARGE { #3 } };
	\foreach \val in {#4}
	{
		\ifx\val\reallyempty
		\else
			\PrintCharWithIndex {#1 + \arabic{x}} {#2} {\val} {\arabic{PrintString@Index}}
			\addtocounter {PrintString@Index} {1}
%			\draw (#1 + \arabic{x},#2) grid +(1, 1);
			\draw[black] (#1 + \arabic{x},#2) rectangle (#1 + \arabic{x} + 1,#2 + 1);
		\fi
		\addtocounter {x} {1}
	}
%	\draw (#1,#2) grid +(\arabic{x}, 1);
}
\newcounter{tokcnt}
\newcommand {\PrintVString} [3]
{
	\setcounter {x} {0}
	\strlenstore{#3}{tokcnt}
	\foreach \val in {1,2,...,\value{tokcnt}}
	{
		\FPmul{\ofsY}{\val}{.7}
		\PrintChar {#1} {#2 - 0.7 + \ofsY} {\color{NumberColor}\substr{#3}{\val}{1}}
	}
}

%
% Enhanced Suffix Array
%
\newcommand{\ltreeexample}
{
		\tikzstyle{lvalue}=[text=black,inner sep=6pt, pos=0.5, font=\bfseries, font=\sf,anchor=center]

		\tikzstyle{siblA}=[sibling distance=50mm, level distance=25mm]
		\tikzstyle{siblB}=[sibling distance=45mm]
		\tikzstyle{siblC}=[sibling distance=25mm]

		\node[inode](r) {[0..10)}
			child [siblA,grow=left] {node (0) at (-3,0) {} edge from parent[draw=none]}
%			child {node (0) {} edge from parent[draw=green]}
%			child {
%				node[inode](1) {[0,0]}
%			}
			child [siblA] {
				node[inode](a) {[0..2)}
				child [siblC] { node[inode](a1) {[0..1)} }
				child [siblC] { node[inode](atctctta1) {[1..2)} }
			}
			child [siblA] {
				node[inode](ct) {[2..4)}
				child [siblC] { node[inode](ctctta1) {[2..3)} }
				child [siblC] { node[inode](ctta1) {[3..4)} }
			}
			child [siblA] {
				node[inode](t) at (3,0) {[4..10)}
				child [siblB] { 
					node[inode](ta) {[4..6)}
					child [siblC] { node[inode](ta1) {[4..5)} }
					child [siblC] { node[inode](tatctctta1) {[5..6)} }
				}
				child [siblB] { 
					node[inode](tct) {[6..8)} 
					child [siblC] { node[inode](tctctta1) {[6..7)} }
					child [siblC] { node[inode](tctta1) {[7..8)} }
				}
				child [siblB] { 
					node[inode](tta) {[8..10)} 
					child [siblC] { node[inode](tta1) {[8..9)} }
					child [siblC] { node[inode](ttatctctta1) {[9..10)} }
				}
			}
			child [siblA, grow=right] {node (0b) at (3,0) {} edge from parent[draw=none]}
		;
		\path (r) -- (0) node [lvalue](l0) {0};
		\path (r) -- (0b) node [lvalue](l11) {10};
%		\path (1) -- (a) node [lvalue](l1) {1};
		\path (a) -- (ct) node [lvalue](l3) {2};
		\path (ct) -- (t) node [lvalue](l5) {4};
		\path (a1) -- (atctctta1) node [lvalue](l2) {1};
		\path (ctctta1) -- (ctta1) node [lvalue](l4) {3};

		\path (ta) -- (tct) node [lvalue](l7) {6};
		\path (tct) -- (tta) node [lvalue](l9) {8};

		\path (ta1) -- (tatctctta1) node [lvalue](l6) {5};
		\path (tctctta1) -- (tctta1) node [lvalue](l8) {7};
		\path (tta1) -- (ttatctctta1) node [lvalue](l10) {9};
		
%		\draw[down] (l0) to node {$\down$} (l1);
%		\draw[nextl] (l1) to node {$\nextl$} (l3);
%		\draw[down] (l1) to node {$\down$} (l2);
%		\draw[nextl] (l3) to node {$\nextl$} (l5);
%		\draw[up] (l3) to node {$\up$} (l2);
%		\draw[down] (l3) to node {$\down$} (l4);
%		\draw[up] (l5) to node {$\up$} (l4);
%		\draw[down] (l5) to node {$\down$} (l7);
%		\draw[nextl] (l7) to node {$\nextl$} (l9);
%		\draw[up] (l7) to node {$\up$} (l6);
%		\draw[down] (l7) to node {$\down$} (l8);
%		\draw[up] (l9) to node {$\up$} (l8);
%		\draw[down] (l9) to node {$\down$} (l10);

%		\draw[nextl] (l0) to node {} (l3);
%		\draw[downS] (l0) to node {} (l2);
		\draw[downS] (l0) to node {} (l3);
		\draw[nextl] (l0) to node {} (l11);
		\draw[nextl] (l3) to node {} (l5);
		\draw[up] (l3) to node {} (l2);
		\draw[downS] (l3) to node {} (l4);
%		\draw[nextl] (l5) to node {} (l11);
		\draw[up] (l5) to node {} (l4);
		\draw[down] (l5) to node {} (l7);
		\draw[nextl] (l7) to node {} (l9);
		\draw[up] (l7) to node {} (l6);
		\draw[downS] (l7) to node {} (l8);
		\draw[up] (l9) to node {} (l8);
		\draw[down] (l9) to node {} (l10);
%		\draw[up] (l11) to node {} (l7);
		\draw[up] (l11) to node {} (l3);
}

\newcommand\tabnode[2]{\tikz \node (#1) {#2};}
\newcommand\tabemptynode[1]{\tikz \node[transparent,outer sep=2pt] (#1) {8};}


%
% Banded Myers
%
\newcommand{\parallelogram}[4] { % params: start x, start y, end x, diagonal width
  \fill[color=black, opacity=0.2] (#1,#2) -- (#3,#2+#3-#1) -- (#3,#2+#3-#1+#4) -- (#1,#2+#4) -- cycle;
}
\newcommand{\pix}[3] { \draw[fill=#3,color=#3] (#1,#2) circle (2pt); }
\newcommand{\diagline}[3] {
  \draw[line width=1pt](#1+.5,#2+.5)--(#1+#3+.5,#2+#3+.5);
}
\newcommand{\diaglineDotted}[3] {
  \draw[line width=0.8pt,densely dotted](#1+.5,#2+.5)--(#1+#3+.5,#2+#3+.5);
}
\newcommand{\diaglineB}[3] {
  \draw[line width=1pt](#1+.5,#2+.5)--(#1+#3+.5,#2+#3+.5);
}
\newcommand{\horizline}[3] {
  \draw[line width=1pt](#1+.5,#2+.5)--(#1+#3+.5,#2+.5);
}
\newcommand{\vertline}[3] {
  \draw[line width=1pt](#1+.5,#2+.5)--(#1+.5,#2+#3+.5);
}
\newcommand{\bandedDotPara}[4]
{ %x,y,d,w
	\foreach \val in {1,...,#4}
	{
		\ifnum\val<#3
			\fill[fill=DPRecurseColor] (#1+\val,#2+1) rectangle (#1+\val + 1, #2+\val + 1);
		\else
			\ifnum\val<\numexpr #4-1
				\fill[fill=DPRecurseColor] (#1+\val,#2+\val-#3+1) rectangle (#1+\val+1,#2+\val+1);
			\else
				\fill[fill=DPRecurseColor] (#1+\val,#2+\val-#3+1) rectangle (#1+\val+1,#2+#4-1);
			\fi
		\fi
	}
}

\newcommand{\bandedStepsBackground}
{
	% paralelogram and grid
	\begin{pgfscope}
		\tikzstyle{grid lines}=[gray,densely dotted,line width=0.3pt]
		\pgftransformshift{\pgfpoint{-3cm}{0cm}}
		\clip (3,-7.3) rectangle (6.3, 7.3);
		\bandedDotPara{0}{0}{11}{19};
		\fill[fill=DPInitColor] (3,0) rectangle (4,4);
		\fill[fill=DPInitColor] (3,0) rectangle (11,1);
%		\draw[black, densely dotted] (12,-1) rectangle (13, 20);
		\draw[style=grid lines](0,0) grid +(20,20);
		\draw[gray] (3,0) rectangle (19, 13);
	\end{pgfscope}
}

\newcommand{\bandedInitBackground}
{
	% paralelogram and grid
	\begin{pgfscope}
		\tikzstyle{grid lines}=[gray,densely dotted,line width=0.3pt]
		\pgftransformshift{\pgfpoint{-3cm}{0cm}}
		\clip (3,-7.3) rectangle  (15.3, 7.3);
		\bandedDotPara{0}{0}{11}{19};
		\fill[fill=DPInitColor] (3,-7) rectangle (4,-6);
		\fill[fill=DPInitColor] (3,-6) rectangle (5,-5);
		\fill[fill=DPInitColor] (3,-5) rectangle (6,-4);
		\fill[fill=DPInitColor] (3,-4) rectangle (7,-3);
		\fill[fill=DPInitColor] (3,-3) rectangle (8,-2);
		\fill[fill=DPInitColor] (3,-2) rectangle (9,-1);
		\fill[fill=DPInitColor] (3,-1) rectangle (10,0);
		\fill[fill=DPInitColor] (3,0) rectangle (11,1);
		\fill[fill=DPInitColor] (3,0) rectangle (4,4);
		\fill[fill=DPInitColorDark] (3,3) rectangle (4,4);
		\fill[fill=DPRecurseColorDark] (4,4) rectangle (5,5);
		\fill[fill=DPRecurseColorDark] (5,5) rectangle (6,6);
		\fill[fill=DPRecurseColorDark] (6,6) rectangle (7,7);
		\fill[fill=DPRecurseColorDark] (7,7) rectangle (8,8);
		\fill[fill=DPRecurseColorDark] (8,8) rectangle (9,9);
		\fill[fill=DPRecurseColorDark] (9,9) rectangle (10,10);
		\fill[fill=DPRecurseColorDark] (10,10) rectangle (11,11);
		\fill[fill=DPRecurseColorDark] (11,11) rectangle (12,12);
		\fill[fill=DPRecurseColorDark] (12,12) rectangle (19,13);
%		\draw[black, densely dotted] (12,-1) rectangle (13, 20);
		\draw[style=grid lines](0,-7) grid +(20,20);
		\draw[black] (3,0) rectangle (19, 13);
	\end{pgfscope}
}
