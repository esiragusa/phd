% ----------------------------------------------------------
% Abbreviations
% ----------------------------------------------------------

\newcommand{\ie}{i.e.\@\xspace}
\newcommand{\eg}{e.g.\@\xspace}
\newcommand{\Eg}{E.g.\@\xspace}
\newcommand{\ea}{\emph{et~al.}\@\xspace}
\newcommand{\wrt}{w.r.t.\xspace}
\newcommand{\st}{s.t.\@\xspace}
\newcommand{\wlogs}{w.l.o.g.\xspace}
\newcommand{\iid}{i.i.d.\@\xspace}

% ----------------------------------------------------------
% Math
% ----------------------------------------------------------

\renewcommand{\emptyset}{\varnothing}
\renewcommand{\min}{\text{min}}
\renewcommand{\max}{\text{max}}
\DeclareMathOperator*{\argmin}{arg\,min}
\DeclareMathOperator*{\argmax}{arg\,max}

\newcommand\norm[1]{\left\lVert#1\right\rVert}
\newcommand\xceil[1]{\left \lceil #1 \right \rceil }
\newcommand\xfloor[1]{\left \lfloor #1 \right \rfloor }

\makeatletter
\def\imod#1{\allowbreak\,{\operator@font mod}\,#1}
\makeatother

\newcommand\bigforall{\mbox{\Large $\mathsurround0pt\forall$}} 

\newcommand{\NP}{\textsl{NP}\xspace}

\newcommand{\sent}{\texttt{\$}\xspace}
\xspaceaddexceptions{\sent}

\newcommand{\Oh}{\mathcal{O}}
%\newcommand{\M}{\mathcal{M}}
%\newcommand{\F}{\ensuremath{\mathbb {F}}}
\newcommand{\Bo}{\ensuremath{\mathbb {B}}}
\newcommand{\N}{\ensuremath{\mathbb {N}}}
\newcommand{\R}{\ensuremath{\mathbb {R}}}
\renewcommand{\P}{\ensuremath{\mathcal {P}}}
\newcommand{\Z}{\ensuremath{\mathbb {Z}}}
\newcommand{\Q}{\ensuremath{\mathbb {Q}}}
\newcommand{\E}{\ensuremath{\mathbb {E}}}
\renewcommand{\S}{\ensuremath{\mathbb {S}}}
\renewcommand{\T}{\ensuremath{\mathbb {T}}}
\newcommand{\checked}{\checkmark}
\newcommand{\TReg}{\textsuperscript*{\textregistered}\xspace}
\newcommand{\TCop}{\textsuperscript*{\textcopyright}\xspace}
\newcommand{\TTra}{\textsuperscript*{\texttrademark}\xspace}

% ----------------------------------------------------------
% Basic notations
% ----------------------------------------------------------

\newcommand\Ibase{0}								% indices 0- or 1-based

\newcounter{At}
\newcommand\At[1]{\setcounter{At}{#1}\addtocounter{At}{-1}\arabic{At}}	% Convert 1-based index to 0-based
\newcommand\AtEnd[1]{#1-1}
\newcommand\LeqThan[1]{< #1}						% Convert 1-based <= to <

\newcommand\Interval[2]{\left[#1,#2\right]}			% [] interval
\newcommand\OInterval[2]{\left[#1,#2\right)}		% [) interval

\newcommand\SAInterval[2]{\OInterval{#1}{#2}}		% SA-interval

%\newcommand\Subscript[2]{#1[#2]}					% string subscript S[i]
\newcommand\Subscript[2]{\MakeLowercase{#1}_{#2}}	% string subscript s_i
\newcommand\SubscriptM[3]{\MakeLowercase{#1}_{#2,#3}} % matrix subscript s_{i,j}
\newcommand\SubscriptNV[2]{#1\left[#2\right]}		% numerical vector subscript V[i]
\newcommand\SubscriptNM[3]{#1\left[#2,#3\right]}	% numerical matrix subscript M[i,j]

%\newcommand\Substring[3]{#1\Interval{#2}{#3}}   	% substring S[i,j]
%\newcommand\OSubstring[3]{#1\OInterval{#2}{#3}} 	% substring S[i,j)
%\newcommand\Substring[3]{{#1}_{\Interval{#2}{#3}}}   	% substring s_[i,j]
%\newcommand\OSubstring[3]{{#1}_{\OInterval{#2}{#3}}} 	% substring s_[i,j)
\newcommand\Substring[3]{{#1}_{{#2}\dots{#3}}}   	% substring S_{i..j}
\newcommand\OSubstring[3]{{#1}_{{#2}\dots{#3-1}}} 	% substring S_{i..j-1}

%\newcommand\SubstringNV[3]{{#1}\Interval{#2}{#3}}   % numerical subvector V[i,j]
%\newcommand\OSubstringNV[3]{{#1}\OInterval{#2}{#3}} % numerical subvector V[i,j)
\newcommand\SubstringNV[3]{{#1}\left[{#2}\dots{#3}\right]} % numerical subvector V[i..j]

\newcommand\Prefix[2]{\Substring{#1}{\Ibase}{#2}}	% prefix as substring
\newcommand\Suffix[2]{{#1}_{#2}}					% suffix S_i

\newcommand\AtCollection[2]{{#1}^{#2}}				% S^i as i-th string in collection

\newcommand\Rev[1]{\bar{#1}}

% ----------------------------------------------------------
% Approximate string matching notations
% ----------------------------------------------------------

\newcommand\String{S}
\newcommand\Strings{\mathbb{\String}}
\newcommand\XSeq{X}
\newcommand\YSeq{Y}
\newcommand\Xlen{n}
\newcommand\Ylen{m}
\newcommand\Slen{n}
\newcommand\Scard{c}
\newcommand\Text{T}
\newcommand\Texts{\mathbb{\Text}}
\newcommand\Tlen{n}
\newcommand\Pattern{P}
\newcommand\Patterns{\mathbb{\Pattern}}
\newcommand\Plen{m}
\newcommand\Database{\mathbb{D}}

\newcommand\Genome{G}
\newcommand\Read{R}

\newcommand\First{F}
\newcommand\Last{L}

\newcommand\RtrieOf[1]{RT(#1)}
\newcommand\StrieOf[1]{ST(#1)}
\newcommand\StrieLeaf[1]{#1}
\newcommand\StrieNode{v}
\newcommand\StrieChar{c}
\newcommand\Sarray{A}
\newcommand\Qdir{D}

\newcommand\FunctionName[1]{\textsc{#1}}
\newcommand\FunctionCall[2]{\FunctionName{#1}($#2$)}

\newcommand{\lexrank}[1]{\ifthenelse{\equal{#1}{}}{\text{rank}}{\text{rank}(#1)}}
\newcommand{\lexlt}{<_{lex}}
\newcommand{\lexleq}{\leq_{lex}}
\newcommand{\lexmin}[1]{\min_{lex}{#1}}
\newcommand{\lexnext}[1]{\text{next}_{lex}{#1}}


\newcommand{\rank}{\FunctionName{rank}}
\newcommand{\popcount}{\FunctionName{popcount}}

\newcommand\Threshold{t}
\newcommand\Weight[1]{w(#1)}
\newcommand\Span[1]{s(#1)}

\newcommand\Stratum{\mathbb{S}}

\newcommand\Scheme{T}
%\newcommand\Transcript{\mathbf{x}}
\newcommand\Transcript{X}
\newcommand\Detected{\mathbb{D}}

% ----------------------------------------------------------
% Tables
% ----------------------------------------------------------

\newcommand\subcolbeg{\setlength{\extrarowheight}{.0ex}\renewcommand{\tabcolsep}{1pt}\tiny}
\newcommand\subcolend{\setlength{\extrarowheight}{.4ex}}
\newcommand\subcolvspace{\vspace{.02ex}}
\setlength{\extrarowheight}{.35ex}

% ----------------------------------------------------------
% Algorithms
% ----------------------------------------------------------

\renewcommand{\And}{\textbf{and }}
\newcommand{\Or}{\textbf{or }}
\newcommand{\NotB}{\textbf{not }}
\newcommand{\To}{\textbf{to }}
\newcommand{\Report}{\textbf{report }}
\newcommand{\Inc}{\textbf{increment }}
\newcommand{\False}{\textbf{false }}
\newcommand{\True}{\textbf{true }}
\newcommand{\CC}{C\raise.06ex\hbox{\tt ++}\xspace}
\newcommand{\Algorithm}[2]{\caption{\textsc{#1}(#2)}}
\renewcommand\algorithmicthen{}

\algdef{SE}[DOWHILE]{Do}{DoWhile}{\algorithmicdo}[1]{\algorithmicwhile\ #1}%

\algdef{SE}[TIMES]{DoRepeat}{Times}{\algorithmicrepeat}[1]{\ #1 \textbf{ times}}%

% ----------------------------------------------------------
% Checks
% ----------------------------------------------------------

\newcommand{\cmark}{\ding{51}}
\newcommand{\xmark}{\ding{55}}
\newcommand{\bmark}{$\bullet$}
