\section*{Abstract}
\label{sec:abstract:en}

Over the last years, high-throughput sequencing (HTS) has become an invaluable method of investigation in molecular  and medical biology.
HTS technologies allow to sequence cheaply and rapidly an individual's DNA sample under the form of billions of short DNA reads.
The ability to assess the content of a DNA sample at base-level resolution opens to a myriad of applications, including individual genotyping and assessment of large structural variations, measurement of gene expression levels and characterization of epigenetic features.
Nonetheless, the quality and quantity of data produced by HTS instruments call for computationally accurate and efficient analysis methods.

In this thesis, I present novel methods for the \emph{efficient} and \emph{accurate} mapping of high-throughput sequencing DNA reads, based on state of the art \emph{approximate string matching} algorithms and data structures.
\emph{Read mapping} is a fundamental step of any HTS data analysis pipeline in resequencing projects, where DNA reads are reassembled by aligning them back to a previously known reference genome.
The ingenuity of approximate string matching methods is crucial for the design of read mapping tools.

In the first part of this thesis, I cover practical \emph{indexed} and \emph{filtering} methods for exact and approximate string matching.
I expose state of the art algorithms and data structures, give their pseudocode and discuss their implementation.
Furthermore, I provide all implementations within SeqAn, the generic \CC template library for sequence analysis, which is freely available under \url{http://www.seqan.de/}.
Subsequently, I experimentally evaluate all implemented methods, with the aim of guiding the engineering of new sequence alignment software.
To the best of my knowledge, this is the first work providing a comprehensive exposition, implementation and evaluation of such methods.

In the second part of this thesis, I turn to the engineering and evaluation of read mapping tools.
First, I present a novel method to find all mapping locations per read within a user-defined error rate.
This method is published in the peer-reviewed journal \emph{Nucleic Acids Research} and packaged in a open source tool nicknamed \emph{Masai}.
Afterwards, I present \emph{Yara}, refining a refinement of Masai's method to quickly report all read mapping locations \emph{stratified} by error rate.
The extensive evaluation, both on simulated and real datasets, shows that Masai and Yara have better speed and accuracy than de-facto standard read mapping tools.


\newpage
