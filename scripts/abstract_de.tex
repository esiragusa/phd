\section*{Zusammenfassung}
\label{sec:abstract:de}

\selectlanguage{ngerman}

In den letzten Jahren ist die Hochdurchsatz-Sequenzierung (HTS) zu einem unverzichtbaren Bestandteil von Molekulare Medizinische Forschung geworden. 
HTS Technologien erm"oglichen DNA-Probe von Einzelpersonen in Form von Milliarden kurze DNA-Reads schnell und g"unstig zu Sequenzieren.
Die F"ahigkeit, die Gehalt einer DNA-Probe mit Aufl"osung von einzige Basenpaare zu bewerten, er"offnet viele neue Anwendungsgebiete, wie die Genotypisierung, die Beurteilung von strukturelle Variationen, die Messung der Genexpressionsleveln und die Etablierung von Epigenetische Faktoren.
Dennoch, die Menge und Qualit"at von HTS Daten ben"otigen effiziente und genaue Analyse durch komputationellen Methoden.
%Over the last years, high-throughput sequencing (HTS) has become an invaluable method of investigation in molecular and medical biology.
%HTS technologies allow to sequence cheaply and rapidly an individual's DNA sample under the form of billions of short DNA reads.
%The ability to assess the content of a DNA sample at base-level resolution opens to a myriad of applications, including individual genotyping and assessment of large structural variations, measurement of gene expression levels and characterization of epigenetic features.
%Nonetheless, the quantity and quality of data produced by HTS instruments call for computationally efficient and accurate analysis methods.

Diese Arbeit pr"asentiert neuartige Methoden f"ur die Mapping von HTS DNA-Reads, die auf den neuesten Stand der Datenstrukturen und Algorithmen f"ur Unscharfe-Stringsuche basierte sind.
Reads Mapping ist ein entscheidender Schritt von jede Datenanalyse-Pipeline im Resequenzierungsprojekten, wo die DNA-reads durch Alignierung auf ein bekannten Referenzgenom z"uruck assembliert werden.
Der Einfallsreichtum von Unscharfe-Stringsuche Methoden ist entscheidend f"ur die Entwicklung von genaue und effiziente Read Mapping Programmen.
%In this thesis, I present novel methods for the mapping of high-throughput sequencing DNA reads, based on state of the art approximate string matching algorithms and data structures.
%Read mapping is a fundamental step of any HTS data analysis pipeline in resequencing projects, where DNA reads are reassembled by aligning them back to a previously known reference genome.
%The ingenuity of approximate string matching methods is crucial to design efficient and accurate read mapping tools.

Die erster Teil dieser Arbeit geht um praktische Indizierung und Filterungsmethoden f"ur Unscharfe-Stringsuche.
Der neuesten Stand der Datenstrukturen und Algorithmen f"ur Unscharfe-Stringsuche ist mit Pseudocode sowie mit eine Diskussion "uber die Details der Implementierung pr"asentiert.
Die Implementierung ist Teil von \emph{SeqAn}, der generischen \CC Template-Bibliothek f"ur Sequenzanalyse, verf"ugbar unter \url{http://www.seqan.de/}.
Im Anschluss, eine experimentelle Bewertung von alle Methoden wurde durchgef"uhrt, mit dem Ziel, die Entwicklung von neue Sequenzanalyse Programme zu f"uhren.
Nach meinem bestem Wissen und Gewissen, das ist die erste Arbeit, dass umfassende Erkl"arung, Implementierung und Bewertung von solche Methoden bietet.
%In the first part of this thesis, I cover practical indexed and filtering methods for exact and approximate string matching.
%I present state of the art algorithms and data structures, give their pseudocode and discuss their implementation.
%Furthermore, I provide all implementations within \emph{SeqAn}, the generic \CC template library for sequence analysis, which is freely available under \url{http://www.seqan.de/}.
%Subsequently, I experimentally evaluate all implemented methods, with the aim of guiding the engineering of new sequence alignment software.
%To the best of my knowledge, this is the first work providing a comprehensive exposition, implementation and evaluation of such methods.

Die zweiter Teil dieser Arbeit geht um die Entwicklung und Bewertung von Read Mapping Programme.
Zuerst, pr"asentiert wird eine neuartige Methode dass, die DNA-Reads an alle ihren potentiellen genomischen Ursprung aligniert, bis einer spezifizierten Fehlerquote;
diese Methode wird ins Open Source Programm \emph{Masai} implementiert und in der Zeitung \emph{Nucleic Acids Research} publiziert.
Danach, wird die Methode generalisiert um alle Co-optimalen oder Suboptimalen Ursprung effizienter zu finden;
diese zweite Methode, ins Open Source Programm \emph{Yara} implementiert, bietet eine Praktischere doch rechtliche fundierten L"osung der Read Mapping Problem.
Eine eingehende Bewertung auf Simulierten und Realen Daten ergab, dass Yara schneller und genauer als De-facto-Standard Read Mapping Programme ist.
%In the second part of this thesis, I turn to the engineering and evaluation of read mapping tools.
%First, I present a novel method to find all mapping locations per read within a user-defined error rate;
%this method is published in the peer-reviewed journal \emph{Nucleic Acids Research} and packaged in a open source tool nicknamed \emph{Masai}.
%Afterwards, I generalize this method to quickly report all co-optimal or suboptimal mapping locations per read within a user-defined error rate;
%this method, packaged in a tool called \emph{Yara}, provides a more practical, yet sound solution to the read mapping problem.
%Extensive evaluations, both on simulated and real datasets, show that Yara has better speed and accuracy than \emph{de-facto} standard read mapping tools.

\selectlanguage{english}

\newpage
