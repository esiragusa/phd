\section*{Zusammenfassung}
\label{sec:abstract:de}

\selectlanguage{ngerman}

In den letzten Jahren ist die Hochdurchsatz-Sequenzierung (HTS) zu einem unverzichtbaren Bestandteil von Molekulare Medizinische Forschung geworden. 
HTS Technologien erm"oglichen DNA-Probe von Einzelpersonen in Form von Milliarden kurze DNA-Reads schnell und g"unstig zu Sequenzieren.
Die F"ahigkeit, die Gehalt einer DNA-Probe mit Aufl"osung von einzige Basenpaare zu bewerten, er"offnet viele neue Anwendungsgebiete, wie die Genotypisierung, die Beurteilung von strukturelle Variationen, die Messung der Genexpressionsleveln und die Etablierung von Epigenetische Faktoren.
Dennoch, die Menge und Qualit"at von HTS Daten ben"otigen effiziente und genaue Analyse durch komputationellen Methoden.

Diese Arbeit pr"asentiert neuartige Methoden f"ur die Mapping von HTS DNA-Reads, die auf den neuesten Stand der Datenstrukturen und Algorithmen f"ur approximative Stringsuche basierte sind.
Reads Mapping ist ein entscheidender Schritt von jede Datenanalyse-Pipeline im Resequenzierungsprojekten, wo die DNA-reads durch Alignierung auf ein bekannten Referenzgenom z"uruck assembliert werden.
Der Einfallsreichtum von approximative Stringsuche Methoden ist entscheidend f"ur die Entwicklung von genaue und effiziente Read Mapping Programmen.

Die erster Teil dieser Arbeit geht um praktische Indizierung und Filterungsmethoden f"ur approximative Stringsuche.
Der neuesten Stand der Datenstrukturen und Algorithmen f"ur approximative Stringsuche ist mit Pseudocode sowie mit eine Diskussion "uber die Details der Implementierung pr"asentiert.
Die Implementierung ist Teil von \emph{SeqAn}, der generischen \CC Template-Bibliothek f"ur Sequenzanalyse, verf"ugbar unter \url{http://www.seqan.de/}.
Im Anschluss, eine experimentelle Bewertung von alle Methoden wurde durchgef"uhrt, mit dem Ziel, die Entwicklung von neue Sequenzanalyse Programme zu f"uhren.
Nach meinem bestem Wissen und Gewissen, das ist die erste Arbeit, dass umfassende Erkl"arung, Implementierung und Bewertung von solche Methoden bietet.

Die zweiter Teil dieser Arbeit geht um die Entwicklung und Bewertung von Read Mapping Programme.
Zuerst, pr"asentiert wird eine neuartige Methode dass, die DNA-Reads an alle ihren potentiellen genomischen Ursprung aligniert, bis einer spezifizierten Fehlerquote;
diese Methode wird ins Open Source Programm \emph{Masai} implementiert und in der Zeitung \emph{Nucleic Acids Research} publiziert.
Danach, wird die Methode generalisiert um alle Co-optimalen oder Suboptimalen Ursprung effizienter zu finden;
diese zweite Methode, ins Open Source Programm \emph{Yara} implementiert, bietet eine Praktischere doch rechtliche fundierten L"osung der Read Mapping Problem.
Eine eingehende Bewertung auf Simulierten und Realen Daten ergab, dass Yara schneller und genauer als De-facto-Standard Read Mapping Programme ist.

\selectlanguage{english}

\newpage
