\section*{Zusammenfassung}
\label{sec:abstract:de}

\selectlanguage{ngerman}

In den letzten Jahren wurde die Hochdurchsatz-Sequenzierung (HTS) zu einem unverzichtbaren Bestandteil der molekularmedizinischen Forschung. 
HTS Technologien erm"oglichen es DNS-Proben von Individuuen schnell und g"unstig in Form von Milliarden kurzer DNS-Reads zu sequenzieren.
Die F"ahigkeit, die Basenpaarabfolge einer DNS-Probe zu bestimmen, er"offnet viele neue Anwendungsgebiete, wie zum Beispiel die Genotypisierung, die Beurteilung von strukturellen Variationen, die Messung der Genexpressionslevel oder die Etablierung von epigenetischen Faktoren.
Jedoch setzen sowohl Quantit"at als auch die Qualit"at der von HTS-Technologien produzierten Daten recheneffiziente und akkurate Analysemethoden voraus um als Standardverfahren im biomedizinischen Bereich eingesetzt werden zu k"onnen.

In dieser Arbeit stelle ich neue Methoden f"ur das sogenannte Read Mapping von HTS Daten vor.
Read Mapping ist ein essentielles Verfahren, bei dem aus den Produkten einer HTS Applikation mit Hilfe eines bereits bekannten Referenzgenoms die urspr"ungliche DNS-Reads assembliert wird. 
Die Entwicklung von speziellen und neuartigen Algorithmen f"ur die approximative Stringsuche spielt dabei eine entscheidende Rolle, um akkurate und effiziente Read Mapping Programme zu entwicklen.

Im ersten Teil dieser Arbeit beschreibe ich praktische Index-Datenstrukturen sowie Filtermethoden, die bei der approximativen Stringsuche angewendet werden. Ich stelle die Algorithmen im Pseudocode dar und bespreche deren Funktionsweise und Implementierung im Detail.
S"amtliche Algorithmen und Datenstrukturen, die ich in dieser Arbeit vorstelle, wurden in \emph{SeqAn} - einer generischen \CC Template-Bibliothek f"ur Sequenzanalyse - implementiert und sind dar"uber verf"ugbar (siehe  \url{http://www.seqan.de/}).
Anschlie{\ss}end analysiere und vergleiche ich die vorgestellten Methoden in verschiedenen Experimenten ausf"uhrlich miteinander, mit dem Ziel neue verbesserte Alignmentalgorithmen entwerfen zu k"onnen.
Nach meinem besten Wissen und Gewissen, ist dies die erste Arbeit, die die genannten Methoden umfassend in ihrer Implementierung und Funktionsweise diskutiert und bewertet.

Im zweiten Teil dieser Arbeit beschreibe und diskutiere ich zwei neue Read Mapping Programme und vergleiche diese mit bisherigen Anwendungen.
Dabei stelle ich zuerst eine neue Methode vor, welche alle potentiellen genomischen Urpsprungspositionen f"ur jeden DNS-Read mit einer spezifizierten Fehleranzahl lokalisiert.
Die beschriebene Methode wurde im Open Source Programm \emph{Masai} implementiert und wurde in der Zeitschrift \emph{Nucleic Acids Research} publiziert.
Anschlie{\ss}end generalisiere ich die Methode und zeige wie alle co-optimalen oder suboptimalen Positionen, gegeben einer vom Nutzer definierten Fehlerrate, effizient gefunden werden k"onnen.
Das Open Source Programm \emph{Yara} implementiert dieses Verfahren und bietet somit eine wesentlich praktischere und noch dazu solide L"osung f"ur das Read Mapping Problem.
Eine umfassende Analyse auf simulierten und realen Daten ergab, dass Yara schneller und genauer als De-facto-Standard Read Mapping Programme ist.

\selectlanguage{english}

\newpage
