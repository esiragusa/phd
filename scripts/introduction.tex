\chapter{Introduction}

% -----------------------------------------------------------------------------

\section{Motivation}

%This work has been motivated by recent advances of molecular genetics.
%The human genome has been sequenced in 2001. Also mouse, drosophila, etc.
%Nowadays \# reference model genomes are available in genbank.
%
%Next-generation sequencing has been the second revolution.
%NGS produces billions of reads for 1000\$ dollars.
%Why should one re-sequence a known genome?
%Resequencing applications include variant calling, etc.
%So NGS impacts biomedicine.
%
%Given a set of reads, two approaches are possible: assembly and mapping.
%
%Assembly methods are based on overlaps, de brujin graphs, or...
%
%Read mapping methods work on a previously assembled reference genome.
%
%The typical SNPs analysis pipeline~\ref{fig:ngs-pipeline} consists of...
%
%In this work we focus on read mapping, although many core algorithms considered are also applicable to assembly, as well as to later pipeline stages.
%
%\begin{figure}[h]
%\caption{NGS pipeline.}
%\label{fig:ngs-pipeline}
%\end{figure}
%
%\begin{itemize}
%\item Plan A: de-novo assembly
%\item Plan B: reference mapping
%\item Plan C: reference guided de-novo assembly
%\end{itemize}

\section{Outline}

\section{Contributions}