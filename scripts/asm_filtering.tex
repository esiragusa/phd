\chapter{Filtering Methods}

\section{Gapped $q$-grams}

TODO.

%\subsection{Threshold computation}
%\subsection{Sensitivity computation}

\begin{figure}[h]
\begin{center}
\caption{Filtration with gapped $q$-grams.}
\label{fig:qgrams-gapped}
\begin{tikzpicture}[font=\normalsize\sffamily]

\transcript{25}{G/1/G, C/1/C, T/1/T, T/1/T, N/0/A, G/1/G, T/1/T, G/1/G, C/1/C, G/0/A, T/1/T, A/1/A, T/1/T, T/1/T, A/0/G, A/1/A, G/1/C, C/1/C, C/1/C, G/0/A, T/1/T, T/1/T, A/0/T, A/1/A, T/1/T}{5}
\band{25}
\seed{1}{25}{1}

\node[left=0.25cm of read_1] {$p$} ;
\node[left=0.25cm of genome_1] {$t$} ;

% Unaffected q-grams ###-#
\foreach \x in {4,9,14}
{
	\pgfmathtruncatemacro{\y}{1+Mod(\x-1,5)}
	\qgramg{\x}{$\#$, $ $, $\#$, $\#$, $\#$}{\y}{1}
}

% Covered q-grams ###-#
\foreach \x in {1,2,3,5,6,7,8,10,11,12,13,15,16,17,18,19,20,21}
{
	\pgfmathtruncatemacro{\y}{1+Mod(\x-1,5)}
	\qgramg{\x}{$\#$, $ $, $\#$, $\#$, $\#$}{\y}{0}
}

% Strikes over covered q-grams
\foreach \x in {1,2,3,5,6,7,8,10,11,12,13,15,16,17,18,20}
{
	\pgfmathtruncatemacro{\y}{5-Mod(\x-1,5)}
	\draw[strike] (qgram_\x_\y.north) -- (qgram_\x_\y.south) ;
}
\draw[strike] (qgram_21_3.north) -- (qgram_21_3.south) ;
\draw[strike] (qgram_20_4.north) -- (qgram_20_4.south) ;
\draw[strike] (qgram_19_5.north) -- (qgram_19_5.south) ;

\end{tikzpicture}
\end{center}
\end{figure}

\section{Approximate seeds}

TODO.

%Filtration specificity in terms of candidate locations to verify is strongly correlated to seed length.
%Since we want to maximize the length of the shortest seed, we let the minimum seed length be $\lfloor |p|/(k+1) \rfloor$.
%If we want to improve filtration specificity by increasing seed length, we can resort to approximate seeds.
%A more involved filtering algorithm proposed in \citep{Navarro2000} reduces an approximate search into smaller approximate searches.
%We partition $p$ into $s \leq k+1$ non-overlapping seeds.
%According to the pigeonhole principle each approximate occurrence of $p$ in $t$ then contains an approximate occurrence of some seed within distance $\lfloor k/s \rfloor$.
%
%Approximate seeds can be searched via backtracking on $\Ti$.
%We search $(k \bmod{s}) + 1$ seeds within distance $\lfloor k/s \rfloor$ and the remaining seeds within distance $\lfloor k/s \rfloor - 1$.
%To prove full-sensitivity it suffices to see that, if none of the seeds occurs within its assigned distance, the total distance must be at least $s \cdot \lfloor k/s \rfloor + (k \bmod s) + 1 = k + 1$.
%Hence all approximate occurrences of $p$ in $t$ within distance $k$ will be found.

\begin{figure}[h]
\begin{center}
\caption{Filtration with approximate seeds.}
\label{fig:seeds-apx}
\begin{tikzpicture}[font=\normalsize\sffamily]

\transcript{25}{G/M/G, C/M/C, T/M/T, N/R/A, T/M/T, G/M/G, G/D/$-$, G/M/G, C/M/C, A/M/A, T/M/T, T/M/T, A/R/G, T/M/T, G/M/G, G/M/G, C/M/C, $-$/I/C, C/M/C, A/M/A, T/M/T, T/M/T, T/M/T, T/R/A, T/M/T}{1}
\band{25}
\seed{1}{25}{1}

\node[left=0.25cm of read_1] {$p$} ;
\node[left=0.25cm of genome_1] {$t$} ;

% Unaffected seeds
\foreach \x in {9}
{
	\qgramo{\x}{$\#$, $\#$, $\#$, $\#$}{1}{1}
}

% Covered seeds
\foreach \x in {1,5,13,22}
{
	\qgramo{\x}{$\#$, $\#$, $\#$, $\#$}{1}{0}
}
\qgrami{17}{$\#$, $ $, $\#$, $\#$, $\#$}{1}

% Strikes over covered seeds
\draw[strike] (qgram_1_4.north) -- (qgram_1_4.south) ;
\draw[strike] (qgram_5_3.north) -- (qgram_5_3.south) ;
\draw[strike] (qgram_13_1.north) -- (qgram_13_1.south) ;
\draw[strike] (qgram_22_3.north) -- (qgram_22_3.south) ;

\end{tikzpicture}
\end{center}
\end{figure}

\section{Suffix filters}
