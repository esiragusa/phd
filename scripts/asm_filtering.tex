\chapter{Filtering Methods}

\section{Gapped $q$-grams}

TODO.

%\subsection{Threshold computation}
%\subsection{Sensitivity computation}

\begin{figure}[h]
\begin{center}
\caption{Filtration with gapped $q$-grams.}
\label{fig:qgrams-gapped}
\begin{tikzpicture}[font=\normalsize\sffamily]

\transcript{25}{G/1/G, C/1/C, T/1/T, T/1/T, N/0/A, G/1/G, T/1/T, G/1/G, C/1/C, G/0/A, T/1/T, A/1/A, T/1/T, T/1/T, A/0/G, A/1/A, G/1/C, C/1/C, C/1/C, G/0/A, T/1/T, T/1/T, A/0/T, A/1/A, T/1/T}{5}
\band{25}
\seed{1}{25}{1}

\node[left=0.25cm of read_1] {$\Pattern$} ;
\node[left=0.25cm of genome_1] {$\Text$} ;

% Unaffected q-grams ###-#
\foreach \x in {4,9,14}
{
	\pgfmathtruncatemacro{\y}{1+Mod(\x-1,5)}
	\qgramg{\x}{$\#$, $ $, $\#$, $\#$, $\#$}{\y}{1}
}

% Covered q-grams ###-#
\foreach \x in {1,2,3,5,6,7,8,10,11,12,13,15,16,17,18,19,20,21}
{
	\pgfmathtruncatemacro{\y}{1+Mod(\x-1,5)}
	\qgramg{\x}{$\#$, $ $, $\#$, $\#$, $\#$}{\y}{0}
}

% Strikes over covered q-grams
\foreach \x in {1,2,3,5,6,7,8,10,11,12,13,15,16,17,18,20}
{
	\pgfmathtruncatemacro{\y}{5-Mod(\x-1,5)}
	\draw[strike] (qgram_\x_\y.north) -- (qgram_\x_\y.south) ;
}
\draw[strike] (qgram_21_3.north) -- (qgram_21_3.south) ;
\draw[strike] (qgram_20_4.north) -- (qgram_20_4.south) ;
\draw[strike] (qgram_19_5.north) -- (qgram_19_5.south) ;

\end{tikzpicture}
\end{center}
\end{figure}

\section{Approximate seeds}

TODO.

%Filtration specificity in terms of candidate locations to verify is strongly correlated to seed length.
%Since we want to maximize the length of the shortest seed, we let the minimum seed length be $\lfloor |p|/(k+1) \rfloor$.
%If we want to improve filtration specificity by increasing seed length, we can resort to approximate seeds.
%A more involved filtering algorithm proposed in \citep{Navarro2000} reduces an approximate search into smaller approximate searches.
%We partition $p$ into $s \leq k+1$ non-overlapping seeds.
%According to the pigeonhole principle each approximate occurrence of $p$ in $t$ then contains an approximate occurrence of some seed within distance $\lfloor k/s \rfloor$.
%
%Approximate seeds can be searched via backtracking on $\Ti$.
%We search $(k \bmod{s}) + 1$ seeds within distance $\lfloor k/s \rfloor$ and the remaining seeds within distance $\lfloor k/s \rfloor - 1$.
%To prove full-sensitivity it suffices to see that, if none of the seeds occurs within its assigned distance, the total distance must be at least $s \cdot \lfloor k/s \rfloor + (k \bmod s) + 1 = k + 1$.
%Hence all approximate occurrences of $p$ in $t$ within distance $k$ will be found.

\begin{figure}[h]
\begin{center}
\caption{Filtration with approximate seeds.}
\label{fig:seeds-apx}
\begin{tikzpicture}[font=\normalsize]

\tikzstyle{n}=[inner sep=0pt, minimum size=10pt, align=center]
\tikzstyle{e}=[-latex, thin]
\tikzstyle{m}=[draw, shape=circle, clabel, pos=0.4, align=center, inner sep=0pt, minimum size=8pt, font=\tiny]
\tikzstyle{t}=[draw, shape=circle, clabel, pos=0.4, align=center, inner sep=0pt, minimum size=8pt, font=\tiny, fill=LightGray]
\tikzstyle{frame}=[draw, rectangle, thin, inner sep=0pt]
\tikzstyle{covered}=[draw, rectangle, thin, inner sep=0pt, fill=LightGray]
\tikzstyle{tape}=[fill=black]
\tikzstyle{strike}=[-, style=double, ultra thin, decorate, decoration=zigzag]
\tikzstyle{line}=[-, thin]
\tikzstyle{wave}=[-, thin, decorate, decoration={snake, segment length=2.5mm, amplitude=0.4mm}]


\newcommand{\transcript}[2]
{
    \foreach[count=\i] \r/\t/\g in {#2}
    {
    	\node[n] (read_\i) at (0.4*\i,0) {\r};
		\node[n] (genome_\i)  at (0.4*\i,-1) {\g};
		\ifthenelse{\equal{\t}{M}}
	    {
			\draw[e] (read_\i) -- (genome_\i) node[m] (transcript_\i) {\t};
		}
		{
			\draw[e] (read_\i) -- (genome_\i) node[t] (transcript_\i) {\t};
		}
    }
    
    \begin{pgfonlayer}{background} 
		\draw[tape] ([xshift=0.1cm, yshift=-0.01cm]transcript_1.north west) rectangle ([xshift=-0.1cm, yshift=0.01cm]transcript_#1.south east) ;
	\end{pgfonlayer}
}

\newcommand{\seed}[3]
{
	\ifthenelse{\equal{#3}{0}}
    {
%    	\draw[strike] (read_#1.west) -- (read_#2.east) ;

	    \begin{pgfonlayer}{background} 
    		\draw[covered] (read_#1.north west) rectangle (read_#2.south east) ;
		\end{pgfonlayer}
    }
           
	\node[frame] (read_rect) [transform shape, fit = (read_#1) (read_#2)] {};
}

\newcommand{\band}[1]
{
%	\node[frame] (genome_rect) [transform shape, fit = (genome_1) (genome_#1)] {};

	\draw[line] (genome_1.north west) -- (genome_#1.north east) ;
	\draw[line] (genome_1.south west) -- (genome_#1.south east) ;
	\draw[line] (genome_1.north west) -- (genome_1.south west) ;
	\draw[line] (genome_#1.north east) -- (genome_#1.south east) ;
}

\transcript{25}{G/M/G, C/M/C, T/M/T, N/R/A, T/M/T, G/M/G, G/D/$-$, G/M/G, C/M/C, A/M/A, T/M/T, T/M/T, A/R/G, T/M/T, G/M/G, G/M/G, C/M/C, $-$/I/C, C/M/C, A/M/A, T/M/T, T/M/T, T/M/T, T/R/A, T/M/T}
\band{25}

\node[left=0.25cm of read_1] {$x$} ;
\node[left=0.25cm of genome_1] {$y$} ;
\node[left=0.25cm of transcript_1] {$transcript$} ;

% (a) Exact seeds
% GCTN TGGG CATT ATGG C-CAT TTTT
% GCTA TG-G CATT GTGG CCCAT TTAT
%
\seed{1}{4}{0}
\seed{5}{8}{0}
\seed{9}{12}{1}
\seed{13}{16}{0}
\seed{17}{21}{0}
\seed{22}{25}{0}

% (b) Approximate seeds
% GCTNTGGG CATTATGG C-CATTTTT
% GCTATG-G CATTGTGG CCCATTTAT
%
%\seed{1}{8}{0}
%\seed{9}{16}{1}
%\seed{17}{25}{0}

\end{tikzpicture}
\end{center}
\end{figure}

\section{Suffix filters}
